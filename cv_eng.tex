%%%%%%%%%%%%%%%%%%%%%%%%%%%%%%%%%%%%%%%%%
% "ModernCV" CV and Cover Letter
% LaTeX Template
% Version 1.11 (19/6/14)
%
% This template has been downloaded from:
% http://www.LaTeXTemplates.com
%
% Original author:
% Xavier Danaux (xdanaux@gmail.com)
%
% License:
% CC BY-NC-SA 3.0 (http://creativecommons.org/licenses/by-nc-sa/3.0/)
%
% Important note:
% This template requires the moderncv.cls and .sty files to be in the same 
% directory as this .tex file. These files provide the resume style and themes 
% used for structuring the document.
%
%%%%%%%%%%%%%%%%%%%%%%%%%%%%%%%%%%%%%%%%%

%----------------------------------------------------------------------------------------
% PACKAGES AND OTHER DOCUMENT CONFIGURATIONS
%----------------------------------------------------------------------------------------

\documentclass[10pt,a4paper,roman]{moderncv} % Font sizes: 10, 11, or 12; paper sizes: a4paper, letterpaper, a5paper, legalpaper, executivepaper or landscape; font families: sans or roman

\moderncvstyle{casual} % CV theme - options include: 'casual' (default), 'classic', 'oldstyle' and 'banking'
\moderncvcolor{blue} % CV color - options include: 'blue' (default), 'orange', 'green', 'red', 'purple', 'grey' and 'black'

%\usepackage{lipsum} % Used for inserting dummy 'Lorem ipsum' text into the template

\usepackage[scale=0.92]{geometry} % Reduce document margins
%\setlength{\hintscolumnwidth}{3cm} % Uncomment to change the width of the dates column
%\setlength{\makecvtitlenamewidth}{10cm} % For the 'classic' style, uncomment to adjust the width of the space allocated to your name

%----------------------------------------------------------------------------------------
% NAME AND CONTACT INFORMATION SECTION
%----------------------------------------------------------------------------------------

\firstname{Max} % Your first name
\familyname{Bielka} % Your last name

% All information in this block is optional, comment out any lines you don't need
\title{curriculum vitae}
\address{Jägerstraße 30}{Gütersloh, 33330}
\mobile{(+49) 176 21390547}
\email{max.bielka@alumni.fh-aachen.de}
%\homepage{staff.org.edu/~jsmith}{staff.org.edu/$\sim$jsmith} % The first argument is the url for the clickable link, the second argument is the url displayed in the template - this allows special characters to be displayed such as the tilde in this example
\photo[120pt][0.4pt]{pictures/foto_small.jpg} % The first bracket is the picture height, the second is the thickness of the frame around the picture (0pt for no frame)

%----------------------------------------------------------------------------------------

\begin{document}

\makecvtitle % Print the CV title

%----------------------------------------------------------------------------------------
% EDUCATION SECTION
%----------------------------------------------------------------------------------------

%\section{School}

%\cventry{2007}{Civil Service}{Elders Home Hollenberg}{Lohmar}{}{}
%\cventry{2006}{Abitur}{Gymnasium}{Lohmar}{}{}

%\section{Academics}

%\cventry{2009--2014}{Bachelor Informatik(No Diploma)}{Fachhochschule}{Aachen}{}{}
%\cventry{2008--2009}{Bachelor Informatik (No Diploma)}{RWTH}{Aachen}{}{}

\section{Ausbildung}

\cventry{2014-09 -- 2016-01}{IT Specialist for Software Development 3 Year Apprenticeship shortened to 18 months}{BBS GuT}{Trier}{}{}

%----------------------------------------------------------------------------------------
% WORK EXPERIENCE SECTION
%----------------------------------------------------------------------------------------

\section{Work Experience}


%------------------------------------------------

\cventry{2022-02 -- Now}{Software Architect}{\textsc{GEA farming technologies GmbH}}{Bönen}{}{Software Architect
\newline
\begin{itemize}
\item Product Development Milking Systems
\begin{itemize}
\item Interfacing between PO / PM Structures for technical specifications
\item Technical decisions in our code architecture
\item SCRUM Master activities to keep the team cohesive
\item Extended Support to other Teams / Projects in regards to SCRUM
\end{itemize}
\end{itemize}}

\cventry{2021-01 -- 2022-01}{Senior Software Engineer}{\textsc{GEA farming technologies GmbH}}{Bönen}{}{Software Engineering
\newline
\begin{itemize}
\item Product Development Herd Management System
\begin{itemize}
\item Development, Support and Integration our Herd Management Product
\item Full Stack Software Engineering
\item Continuous Integration / Continuous Deployment Jenkins / Azure DevOPS
\end{itemize}
\end{itemize}}

\cventry{2016-11 -- 2020-12}{Software Engineer}{\textsc{Arvato Distribution SCS-IT}}{Gütersloh}{}{Software Engineering
\newline
\begin{itemize}
\item Product Development WCS (Warehouse Control System)
\begin{itemize}
\item Development, Support and Integration our WCS Products
\item Full Stack Software Engineering
\item Warehouse Integration in 26 Warehouses Worldwide
\item Continuous Integration / Continuous Deployment Azure DevOPS
\item Extended Support SCS-IT USA on site
\end{itemize}
\end{itemize}
\begin{itemize}
\item AI Development Udacity Bertelsmann Challenge Course
\begin{itemize}
\item Development of Neural Networks with pytorch
\end{itemize}
\end{itemize}}

\cventry{2016-02 -- 2016-10}{Software Developer}{\textsc{OPC Cardsystems GmbH + Iprolux S.A.R.L.}}{Wasserbillig}{}{Software Development
\newline{}
\begin{itemize}
\item Full Stack Development POS (Point Of Sales) Systeme
\end{itemize}}

\cventry{2014-09 -- 2016-01}{Apprenticeship Fachinformatiker IT Specialist for Software Development}{\textsc{OPC Cardsystems GmbH + Iprolux S.A.R.L.}}{Trier}{}{Main challenges: Project Work
\newline{}
\begin{itemize}
\item Full Stack Development POS (Point Of Sales) Systeme
\end{itemize}}

\cventry{2013-01 -- 2014-01}{Working Student Programming}{\textsc{IngenieurBüro Dr. Plesnik}}{Aachen}{}{Main challenges: Create Cloud Storage solution on basis of Owncloud
\begin{itemize}
\item Planing, UI Design and Execution of a Cloud storage solution
\end{itemize}}

%------------------------------------------------




%----------------------------------------------------------------------------------------
% COMPUTER SKILLS SECTION
%----------------------------------------------------------------------------------------

\section{Skills}

\cvitem{Languages}{C\#, C++, Java, python, typescript, javascript, Delphi}
\cvitem{Concepts}{MVVM, Docker, Kubernetes, Serverless Architecture, SpringBoot, T-SQL, PostgreSQL, Qt4/5, dotNet 4+, dotNet core 2.0+, Unix, WPF, qmake, mingw32, Azure PaaS, Visual Studio code, Visual Studio, Qt Creator, intellij, eclipse}

\clearpage

%----------------------------------------------------------------------------------------
% Projektliste
%----------------------------------------------------------------------------------------

\section{Projektliste}

\cventry{}{Hobby Projekte}{\textsc{Privat}}{}{}{
Auch nach der Arbeit habe ich viel Spaß an privaten Projekten.
Beispielsweise habe ich mich an Heim Automatisierung mit einem Raspberry Pi versucht.
Umgesetzt mit shellscript und python habe ich unter anderem einen daemon geschrieben, der erkennt, wenn mein Handy sich mit dem WLAN verbindet.
Das ist das Signal die Lichter an zu machen und wieder aus zu schalten, wenn das Handy das WLAN verlässt.
}

\cventry{2019-jetzt}{Produktentwicklung Warehouse Control System AGVs}{\textsc{Arvato SCS-IT}}{}{}{
Als erstes Mitglied der WCS Familie wird eine Middleware Entwickelt, um die Kommunikation zwischen Automation Guided Vehicles und ERP Systemen herzustellen.
Dieses Produkt ist in der Implementierungsphase, seit der Konzeptionsphase war meine Rolle technischer Berater.
Die Erfahrung mit Business Units außerhalb von Europa hat hier bereits positiv eingewirkt.
Das Ziel ist die Middleware als Dockerized Service in unserem Kubernetes zu hosten, da bei diesem Projekt Antwortzeiten nicht die höchste Priorität haben im Gegensatz zu Armada.
}

\cventry{2017-jetzt}{Produktentwicklung Warehouse Control System Armada 2.0}{\textsc{Arvato SCS-IT}}{}{}{
Eine Neuentwicklung als Produkt mit revolutionären Ansätzen.
Das Projekt ist als Middleware entwickelt worden und hat bereits 3 größere Projektphasen durchschritten.
Im DevOPS Konzept wurde in dotNet 4.7 + dotNet core 2.1 diverse Projekte als Produkt entwickelt.
In der Zeit von 2017 bis heute haben wir als Team die Implementierung von Armada 2.0 in 26 Warehäusern weltweit betreut und den Produktgedanken gelebt.
Die aktuelle Projektphase ist, den Code auf cloud optimized zu heben. 
}

\cventry{2016-jetzt}{Produktentwicklung Warehouse Control System Armada 1.0}{\textsc{Arvato SCS-IT}}{}{}{
Das bereits bestehende c\# dotNet 4.5 Projekt musste übernommen werden.
Bis zum Start von Armada 2.0 und der Migration der Legacy Warenhäuser müssen neue Anforderungen umgesetzt werden und unterstützt werden.
Das Projekt hat als Middleware zwischen den SAP ERP Systemen und den Fördertechiken der jeweilgen Warenhäusern gedient.
Es wurden unterschiedliche Dienste implementiert, welche Kommunikation für die Fördertechnik zwischen SPS und SAP abgebildet haben.
Meine Aufgabe war hauptsächlich der Operational Support, Weiterentwicklung und Vorantreiben der Ablösung von Armada 1.0}

\cventry{2015-2016}{Server-/Datenbankentwicklung Chipkarten System}{\textsc{OPC/Iprolux}}{}{}{
In der Projektreihe bin ich als Helfer und Vertretung eingetreten, falls die Hauptentwickler nicht verfügbar sind.\newline{}
Die Projekte basieren auf einer Delphi 6 Codebasis und Datenbanken Pervasive SQL v8/v11.
Ich habe bei Bedarf in jedem dieser Projekte änderungen durchgeführt.\newline{}
 -Datenbank Updates/Indizierung für Reporting\newline{}
 -Reporting Anpassungen bei neuen/alten Produktreihen\newline{}
 -SOAP/Kommunikation Anpassungen für neue Features
 -Unterstützung und Support
}

%\cventry{2016-2016}{GdPdU/INSIKA/GoBD}{\textsc{OPC/Iprolux}}{}{}{
%Das Projekt wurde separat vom GdPdU Exporter aufgesetzt.
%Die Rolle im Projekt ist leitend.\newline{}
%Im Allgemeinen bin ich Ansprechpartner und Audit Leiter, falls es Unklarheiten zu Weiterentwicklungen unserer %Systeme gibt und deren Impikationen in der GoBD Vorschrift.
%Code Anpassungen werden z.T. Unterstützend durchgeführt, falls ein Fehler gemacht wurde, oder neue Anforderungen %des Gesetzgebers aufkommen.
%}


\cventry{2015-2016}{Kassenentwicklung neue Produktlinie Chipkarten}{\textsc{OPC/Iprolux}}{}{}{
In die Projektreihe bin ich als Helfer eingetreten und Vertretung, falls der Hauptentwickler nicht verfügbar ist.
Die Projekte wurden mit Delphi XE 7 und Delphi 6 ausgeführt.\newline{}
TouchPOS XE ist eine komplett neue Codebasis auf Delphi XE 7 Basis mit austauschbarer Sqlite Datenbank(FireDAC).
PhotocheckInXE ist eine alte Codebasis, welche visuell erneuert wurde auf Delphi 6 Basis mit einer Pervasive SQL v8 Datenbank Basis.
Teile der Projekte welche ich übernommen haben sind z.B.:\newline{}
  -Thread Entwicklung der Buchungsverarbeitung TouchPOSXE\newline{}
  -Exception Handler Entwicklung auf beiden Codebasen\newline{}
  -EventLogger Entwicklung, da alter Eventlogger auf Windows8+ bei jedem Schreibvorgang einen Flush Befehl abgesetzt hat (Performance probleme)\newline{}
  -Bestehende SOAP Kommunikation Erweiterung und Errorhandling\newline{}
  -Vereinzelte Qualitiy of Life änderungen und Customizing für Kunden\newline{}
}

\cventry{2014-2016}{Kassenentwicklung NEO mit Lochkarten}{\textsc{OPC/Iprolux}}{}{}{
Das Projekt wurde übernommen und in Delphi XE 4 fertiggestellt mit einer SQLite Datenbank fertiggestellt.
Der Zweite Teil zur Kasse ist eine GdPdU Schnittstelle namens CardDiary, welche in C\# .Net 3.5 von mir übernommen wurde, auch mit einer SQLite Datenbank Basis.
Die Projektrolle war und ist leitend.\newline{}
Bei übernahme war das Projekt sehr Wartungsintensiv und nicht geeignet für den Massenmarkt.\newline{}
ca. 30-40\% des Codes musste neu geschrieben werden.
Grössere Milestones umfassen z.B.:\newline{}
  -Verbesserung der Kommunikation mit Kartenleser (COM Schnittstelle - OPC Telegram Basis)\newline{}
  -Berechnungsalgroithmen neu geschrieben, dass mit allen möglichen Einstellungen korrekt berechnet wird\newline{}
  -6-Bit und 8-Bit Karten Arten sind einsetzbar\newline{}
  -Automatisierte Kommunikation Kasse/GdPdU Schnittstelle\newline{}
  -Autonome Reparatur der Datenbank bei fatalen Fehlern zur Verringerung des Support Aufwandes.\newline{}
Die aktuelle Version ist 1.6.9 mit  Erweiterungen in Richtung Gastro Fähigkeit.
}

%\cventry{2014-2016}{Export Tool für GdPdU konforme Kassendaten}{\textsc{OPC/Iprolux}}{}{}{
%Das Projekt wurde übernommen und in Delphi 6 fertiggestellt.
%Die Projektrolle war ausführend und später leitend, inklusive Kommunikation mit einer GdPdU Zertifizierungsstelle %und Verifikation der integrität der Daten.
%Die Version 1.0 für das CCS System ist August 2014 fertig gestellt worden.\newline{}
%Die aktuelle Version ist 2.7.3 und ist mit allen OPC Produkten kompatibel.
%}


\cventry{2013-2014}{Cloudspeicherlösung auf OwnCloud Basis}{\textsc{IngenieurBüro Dr. Plesnik}}{}{}{
Das Projekt wurde in C++ mit Qt 4.8 (opensource) realisiert.
Ausserdem musste einiger Code der Webseite/des CMS in php angepasst werden.\newline{}
Die Projektrolle war (in der Programmierung) leitend, zeitweise mit 2 unterstützenden Mitarbeitern.\newline{}
Der Opensource Code wurde von GitHub gebranched und für Linux, WIndows und Mac kompiliert.
Anpassungen umfassen z.B.:\newline{}
  -Anpassen der Erscheinung\newline{}
  -Informationen im CLient anzeigen, welche eigentlich nur auf der Weboberfläche angezeigt wurden\newline{}
  -Eine Teilungsmöglichkeit der Dateien vom Client aus\newline{}
  -Die Webseite wurde um eine API erweitert, um mit dem Client besser kommunizieren zu können.
}

%\cventry{2011-2012}{Visualisierung von Industrieanwendung auf Leitrechner Basis}{\textsc{Aucos GmbH}}{}{}{
%Das Projekt wurde in C++ mit Qt 4.7 (opensource) realisiert.\newline{}
%Visualisierungen wurden mit Adobe Flash auf Basis der Schaltzeichnungen erstellt.\newline{}
%Die Projektrolle war unterstützend in Programmierung und Visualisierung.
%\newline{}
%}

%----------------------------------------------------------------------------------------
% COVER LETTER
%----------------------------------------------------------------------------------------

% To remove the cover letter, comment out this entire block

% \clearpage

% \recipient{HR Department}{Corporation\\123 Pleasant Lane\\12345 City, State} % Letter recipient
% \date{\today} % Letter date
% \opening{Sehr geehrte Damen und Herren,} % Opening greeting

% \makeletterclosing % Print letter signature

%----------------------------------------------------------------------------------------

\end{document}
