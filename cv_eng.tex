%%%%%%%%%%%%%%%%%%%%%%%%%%%%%%%%%%%%%%%%%
% "ModernCV" CV and Cover Letter
% LaTeX Template
% Version 1.11 (19/6/14)
%
% This template has been downloaded from: 
% http://www.LaTeXTemplates.com
%
% Original author:
% Xavier Danaux (xdanaux@gmail.com)
%
% License:
% CC BY-NC-SA 3.0 (http://creativecommons.org/licenses/by-nc-sa/3.0/)
%
% Important note:
% This template requires the moderncv.cls and .sty files to be in the same 
% directory as this .tex file. These files provide the resume style and themes 
% used for structuring the document.
%
%%%%%%%%%%%%%%%%%%%%%%%%%%%%%%%%%%%%%%%%%

%----------------------------------------------------------------------------------------
% PACKAGES AND OTHER DOCUMENT CONFIGURATIONS
%----------------------------------------------------------------------------------------

\documentclass[10pt,a4paper,roman]{moderncv} % Font sizes: 10, 11, or 12; paper sizes: a4paper, letterpaper, a5paper, legalpaper, executivepaper or landscape; font families: sans or roman

\moderncvstyle{casual} % CV theme - options include: 'casual' (default), 'classic', 'oldstyle' and 'banking'
\moderncvcolor{blue} % CV color - options include: 'blue' (default), 'orange', 'green', 'red', 'purple', 'grey' and 'black'

%\usepackage{lipsum} % Used for inserting dummy 'Lorem ipsum' text into the template

\usepackage[scale=0.92]{geometry} % Reduce document margins
%\setlength{\hintscolumnwidth}{3cm} % Uncomment to change the width of the dates column
%\setlength{\makecvtitlenamewidth}{10cm} % For the 'classic' style, uncomment to adjust the width of the space allocated to your name

%----------------------------------------------------------------------------------------
% NAME AND CONTACT INFORMATION SECTION
%----------------------------------------------------------------------------------------

\firstname{Max} % Your first name
\familyname{Bielka} % Your last name

% All information in this block is optional, comment out any lines you don't need
\title{curriculum vitae}
\address{Jägerstraße 30}{Gütersloh, 33330}
\mobile{(+49) 176 21390547}
\email{max.bielka@alumni.fh-aachen.de}
%\homepage{staff.org.edu/~jsmith}{staff.org.edu/$\sim$jsmith} % The first argument is the url for the clickable link, the second argument is the url displayed in the template - this allows special characters to be displayed such as the tilde in this example
\photo[120pt][0.4pt]{pictures/foto_small.jpg} % The first bracket is the picture height, the second is the thickness of the frame around the picture (0pt for no frame)

%----------------------------------------------------------------------------------------

\begin{document}

\makecvtitle % Print the CV title

%----------------------------------------------------------------------------------------
% EDUCATION SECTION
%----------------------------------------------------------------------------------------

%\section{School}

%\cventry{2007}{Civil Service}{Elders Home Hollenberg}{Lohmar}{}{}
%\cventry{2006}{Abitur}{Gymnasium}{Lohmar}{}{}

%\section{Academics}

%\cventry{2009--2014}{Bachelor Informatik(No Diploma)}{Fachhochschule}{Aachen}{}{}
%\cventry{2008--2009}{Bachelor Informatik (No Diploma)}{RWTH}{Aachen}{}{}

\section{Apprenticeship}

\cventry{2014-09 -- 2016-01}{IT Specialist for Software Development 3 Year Apprenticeship shortened to 18 months}{BBS GuT}{Trier}{}{}

%----------------------------------------------------------------------------------------
% WORK EXPERIENCE SECTION
%----------------------------------------------------------------------------------------

\section{Work Experience}


%------------------------------------------------

\cventry{2023-04 -- Now}{Solutions Architect}{\textsc{Fujitsu Services}}{Hannover}{}{Solutions Architect BACS
\begin{itemize}
\item Solutions Architect for Microsoft Azure Cloud Native Development
\begin{itemize}
\item Lead Architect for SAP on Azure Proof of Concept
\item Cloud Transition from private Cloud environment
\item Cloud native consulting for our customers
\end{itemize}
\end{itemize}}

\cventry{2022-02 -- 2023-03}{Software Architect}{\textsc{GEA farming technologies GmbH}}{Bönen}{}{Software Architect
\begin{itemize}
\item Product Development Milking Systems
\begin{itemize}
\item Interfacing between PO / PM Structures for technical specifications
\item Technical decisions in our code architecture
\item SCRUM Master activities to keep the team cohesive
\item Extended Support to other Teams / Projects 
\end{itemize}
\end{itemize}}

\cventry{2021-01 -- 2022-01}{Senior Software Engineer}{\textsc{GEA farming technologies GmbH}}{Bönen}{}{Software Engineering
\begin{itemize}
\item Product Development Herd Management System
\begin{itemize}
\item Development, Support and Integration our Herd Management Product
\item Full Stack Software Engineering
\item Continuous Integration / Continuous Deployment Jenkins / Azure DevOPS
\end{itemize}
\end{itemize}}

\cventry{2016-11 -- 2020-12}{Software Engineer}{\textsc{Arvato Distribution SCS-IT}}{Gütersloh}{}{Software Engineering
\begin{itemize}
\item Product Development WCS (Warehouse Control System)
\begin{itemize}
\item Development, Support and Integration our WCS Products
\item Full Stack Software Engineering
\item Warehouse Integration in 26 Warehouses Worldwide
\item Continuous Integration / Continuous Deployment Azure DevOPS
\item Extended Support SCS-IT USA on site
\end{itemize}
\end{itemize}
\begin{itemize}
\item AI Development Udacity Bertelsmann Challenge Course
\begin{itemize}
\item Development of Neural Networks with pytorch
\end{itemize}
\end{itemize}}

\cventry{2016-02 -- 2016-10}{Software Developer}{\textsc{OPC Cardsystems GmbH + Iprolux S.A.R.L.}}{Wasserbillig}{}{Software Development
\begin{itemize}
\item Full Stack Development POS (Point Of Sales) Systems
\end{itemize}}

\cventry{2014-09 -- 2016-01}{Apprenticeship Fachinformatiker IT Specialist for Software Development}{\textsc{OPC Cardsystems GmbH + Iprolux S.A.R.L.}}{Trier}{}{Main challenges: Project Work
\begin{itemize}
\item Full Stack Development POS (Point Of Sales) Systems
\end{itemize}}

%------------------------------------------------




%----------------------------------------------------------------------------------------
% COMPUTER SKILLS SECTION
%----------------------------------------------------------------------------------------

\section{Skills}

\cvitem{Languages}{C\#, C++, Java, python, typescript, javascript, Delphi}
\cvitem{Concepts}{MVVM, Docker, Kubernetes, Serverless Architecture, SpringBoot, T-SQL, PostgreSQL, Qt4/5, dotNet 4+, dotNet core 2.0+, Unix, WPF, qmake, mingw32, Azure PaaS, Visual Studio code, Visual Studio, Qt Creator, intellij, eclipse}

\clearpage

%----------------------------------------------------------------------------------------
% Projektliste
%----------------------------------------------------------------------------------------

\section{Projektliste}

\cventry{2019-2023}{Product Development Warehouse Control System AGVs}{\textsc{Arvato SCS-IT}}{}{}{
As the first member of the WCS family, a middleware is being developed to establish communication between Automation Guided Vehicles and ERP systems.
This product is in the implementation phase, and my role since the conception phase has been that of a technical advisor.
Experience with business units outside of Europe has already had a positive impact here.
The goal is to host the middleware as a Dockerized service in our Kubernetes, as response times are not the highest priority for this project, unlike Armada.
}

\cventry{2017-present}{Product Development Warehouse Control System Armada 2.0}{\textsc{Arvato SCS-IT}}{}{}{
A new development as a product with revolutionary approaches.
The project was developed as middleware and has already gone through three major project phases.
In the DevOPS concept, various projects were developed as a product in dotNet 4.7 + dotNet core 2.1.
As a team, we have supported the implementation of Armada 2.0 in 26 warehouses worldwide since 2017 and lived up to the product idea.
The current project phase is to lift the code to cloud optimized.
}

\cventry{2016-present}{Product Development Warehouse Control System Armada 1.0}{\textsc{Arvato SCS-IT}}{}{}{
The existing c# dotNet 4.5 project had to be taken over.
New requirements had to be implemented and supported until the start of Armada 2.0 and the migration of the legacy warehouses.
The project served as middleware between the SAP ERP systems and the conveyor technology of the respective warehouses.
Various services were implemented that mapped communication for the conveyor technology between PLC and SAP.
My main task was operational support, further development, and driving the replacement of Armada 1.0 forward.
}

\cventry{2015-2016}{Server/Database Development Chip Card System}{\textsc{OPC/Iprolux}}{}{}{
In the project series, I joined as a helper and representative in case the main developers were not available.
The projects are based on a Delphi 6 codebase and Pervasive SQL v8/v11 databases.
I made changes to each of these projects as needed.\newline{}
-Database updates/indexing for reporting\newline{}
-Reporting adjustments for new/old product lines\newline{}
-SOAP/communication adjustments for new features\newline{}
-Support and support
}

\cventry{2015-2016}{Cash Register Development New Chip Card Product Line}{\textsc{OPC/Iprolux}}{}{}{
I joined the project series as a helper and representative in case the main developer was not available.
The projects were executed with Delphi XE 7 and Delphi 6.\newline{}
TouchPOS XE is a completely new codebase based on Delphi XE 7 with an exchangeable SQLite database (FireDAC).
PhotocheckInXE is an old codebase that has been visually renewed based on Delphi 6 with a Pervasive SQL v8 database base.
Parts of the projects that I took over include:\newline{}
-Thread development of the booking processing TouchPOSXE\newline{}
-Exception handler development on both codebases\newline{}
-EventLogger development because the old Eventlogger on Windows8+ sent a flush command with every write operation (performance issues)\newline{}
-Existing SOAP communication extension and error handling\newline{}
-Isolated quality of life changes and customization for customers\newline{}
}

%\cventry{2014-2016}{Export Tool für GdPdU konforme Kassendaten}{\textsc{OPC/Iprolux}}{}{}{
%Das Projekt wurde übernommen und in Delphi 6 fertiggestellt.
%Die Projektrolle war ausführend und später leitend, inklusive Kommunikation mit einer GdPdU Zertifizierungsstelle %und Verifikation der integrität der Daten.
%Die Version 1.0 für das CCS System ist August 2014 fertig gestellt worden.\newline{}
%Die aktuelle Version ist 2.7.3 und ist mit allen OPC Produkten kompatibel.
%}


\cventry{2013-2014}{Cloud storage solution based on OwnCloud}{\textsc{IngenieurBüro Dr. Plesnik}}{}{}{
The project was implemented in C++ using Qt 4.8 (open source).
Additionally, some code of the website/CMS had to be adapted in PHP.\newline{}
The project role was leading (in programming), occasionally with 2 supporting employees.\newline{}
The open-source code was branched from GitHub and compiled for Linux, Windows, and Mac.
Adjustments include:\newline{}
-Adjusting the appearance\newline{}
-Displaying information in the client that was originally only displayed on the web interface\newline{}
-The client was given the ability to share files\newline{}
-The website was expanded with an API to better communicate with the client.

%\cventry{2011-2012}{Visualisierung von Industrieanwendung auf Leitrechner Basis}{\textsc{Aucos GmbH}}{}{}{
%Das Projekt wurde in C++ mit Qt 4.7 (opensource) realisiert.\newline{}
%Visualisierungen wurden mit Adobe Flash auf Basis der Schaltzeichnungen erstellt.\newline{}
%Die Projektrolle war unterstützend in Programmierung und Visualisierung.
%\newline{}
%}

%----------------------------------------------------------------------------------------
% COVER LETTER
%----------------------------------------------------------------------------------------

% To remove the cover letter, comment out this entire block

% \clearpage

% \recipient{HR Department}{Corporation\\123 Pleasant Lane\\12345 City, State} % Letter recipient
% \date{\today} % Letter date
% \opening{Sehr geehrte Damen und Herren,} % Opening greeting

% \makeletterclosing % Print letter signature

%----------------------------------------------------------------------------------------

\end{document}
